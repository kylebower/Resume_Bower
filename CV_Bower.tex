\documentclass{resume} 
\usepackage[left=0.75in,top=0.6in,right=0.75in,bottom=0.6in]{geometry} 
\usepackage{hyperref}
\newcommand{\tab}[1]{\hspace{.2667\textwidth}\rlap{#1}}
\newcommand{\itab}[1]{\hspace{0em}\rlap{#1}}
\name{Kyle Bower}
\address{+1\;(604)\;401-4234 \\ kyle.bower@mail.utoronto.ca}

\begin{document}

\begin{rSection}{Education}

\textbf{University of Toronto} \hfill {\normalfont Sep 2018 -- Jul 2023} \\
{\normalfont \textit{Doctor of Philosophy, Mathematics}} \\
\textbf{University of Toronto} \hfill {\normalfont Sep 2017 -- Aug 2018} \\
{\normalfont \textit{Master of Science, Mathematics}} \\
\textbf{University of British Columbia} \hfill {\normalfont Sep 2013 -- Apr 2017}\\
{\normalfont \textit{Bachelor of Science, Honours Physics and Mathematics}}

\end{rSection}

% \begin{rSection}{Career Objective}
%  To work for an organization which provides me the opportunity to improve my skills and knowledge to grow along with the organization objective.
% \end{rSection}

\begin{rSection}{Work Experience}

\begin{rSubsection}{Head Graduate Teaching Assistant}{Sep 2017 -- Present}{University of Toronto}{Toronto, ON}
    \item Developed a script to automatically assign 1,500+ students to tutorial groups. (Python)
    \item Mentored 100+ students each term in weekly tutorials, held office hours, invigilated exams, and graded tests for undergraduate math courses.
    \item Led orientation and regular training for aspiring and new teaching assistants.
\end{rSubsection}

\begin{rSubsection}{Cryptanalyst Summer Intern}{May 2017 -- Aug 2017}{Communications Security Establishment}{Ottawa, ON}
    \item Performed cryptologic research, designed, developed, assessed, and attacked cryptographic algorithms and protocols to support Government of Canada Intelligence and Cyber Defense requirements.
    \item Attended the Selected Areas in Cryptography Summer School at the University of Ottawa to learn about symmetric key, public key, and post-quantum cryptography.
\end{rSubsection}

\begin{rSubsection}{Undergraduate Student Researcher}{May 2016 -- Aug 2016}{Natural Sciences and Engineering Research Council of Canada}{Vancouver, BC}
    \item Conducted independent research on the Min protein system involved in bacterial cell division under the supervision of Dr. Eric Cytrynbaum.
    \item Simulated cell-scale biochemical reactions to capture the natural stochasticity of a biological cell. (Smoldyn)
    \item Employed numerical methods to solve a system of partial differential equations that described the biochemical reactions occurring in the cell. (MATLAB/GNU Octave)
\end{rSubsection}

\begin{rSubsection}{Undergraduate Teaching Assistant}{Sep 2015 -- Apr 2017}{University of British Columbia}{Vancouver, BC}
    \item Invigilated exams and graded tests for undergraduate math courses.
\end{rSubsection}

\begin{rSubsection}{Lifeguard/Swim Instructor}{Nov 2013 -- Apr 2017}{City of Richmond}{Richmond, BC}
    \item Supervised patrons to ensure a safe and enjoyable aquatic environment.
    \item Taught swimming and water safety to students of all ages and abilities.
\end{rSubsection}

\end{rSection}

\begin{rSection}{Technical Skills}

{\normalfont Experience with Python, MATLAB, LaTeX, UNIX, Linux, GitHub.}

\end{rSection}

\begin{rSection}{Conference Presentations}

\begin{rSubsection}{SIAM Lecture Series}{Apr 6, 2023}{University at Buffalo}{Buffalo, NY}
\item An Efficient Method for the Computation of Electrostatic Potentials for Piecewise Constant Conductivities.
\end{rSubsection}

\begin{rSubsection}{Bird's Eye Conference}{Mar 5, 2023}{University of Toronto}{Toronto, ON}
\item Fast Computation of Electrostatic Potentials for Piecewise Constant Conductivities.
\end{rSubsection}

\begin{rSubsection}{University of Toronto Applied Math Meeting}{Nov 25, 2022}{University of Toronto}{Toronto, ON}
\item Iterative Methods and GMRES.
\end{rSubsection}

\begin{rSubsection}{Great Lakes Section Siam Annual Meeting}{Sep 24, 2022}{Wayne State University}{Detroit, MI}
\item Fast Computation of Electrostatic Potentials for Piecewise Constant Conductivities.
\end{rSubsection}

\begin{rSubsection}{Southern Ontario Numerical Analysis Day}{May 3, 2019}{Ontario Tech University}{Oshawa, ON}
\item Geometric Integration of the Outer Solar System with Symplectic Integrators.
\end{rSubsection}

\end{rSection}

\begin{rSection}{Publications}
    \begin{itemize}
        \item K. Bower, K. Serkh, S. Alexakis, A. R. Stinchcombe, \textit{Fast Computation of Electrostatic Potentials for Piecewise Constant Conductivities} (submitted to SIAM Journal on Scientific Computing 2022). arXiv: \url{https://arxiv.org/abs/2205.15354}
        \item K. Bower, S. Alexakis, A. R. Stinchcombe, \textit{Charge Density Analysis of Overlapping Regions} (in progress).
        \item K. Bower, K. Serkh, S. Alexakis, A. R. Stinchcombe, \textit{A Novel Solver for the Inverse Conductivity Problem in the Case of Piecewise Constant Conductivities} (in progress).
    \end{itemize}
\end{rSection}

\begin{rSection}{Honours and Awards}
\textbf{Arts and Science Program Completion Award (15,000 CAD)} \hfill {\normalfont Sep 2022} \\
\textbf{Ontario Graduate Scholarship (15,000 CAD)} \hfill {\normalfont Sep 2017} \\
\textbf{NSERC Undergraduate Student Research Award (5,625 CAD)}\hfill {\normalfont May 2016} 
% \\
% \textbf{BC Passport to Education (1,000 CAD)}\hfill {\normalfont Sep 2013}
\end{rSection}

\begin{rSection}{Professional Development}

\begin{rSubsection}{Mathematical Sciences Research Institute – Summer Graduate School: Integral Equations and Applications}{Jun 6--17, 2022}{}{}
\item A two week long school to introduce graduate students to the systematic study of integral equations; present some of the latest theoretical advancements in the field and open problems; and involve participants in a hands-on discovery lab focused on deriving results about integral operators in two dimensions relevant for both the theoretical and numerical treatment of Integral Equations in two dimensions. 
\end{rSubsection}

\begin{rSubsection}{Mathematical and Computational Methods in Biology Workshop}{May 5--8, 2020}{}{}
\item Workshop bringing together researchers working on topics of importance to biology by means of mathematical and computational methods.
\end{rSubsection}

\begin{rSubsection}{General Assembly – Intro to Data Science in Toronto}{Dec 9, 2019}{}{}
\item Introductory workshop to see how modern businesses are harnessing the power of data to drive innovation.
\end{rSubsection}

\begin{rSubsection}{CMS Winter Meeting}{ Dec 6--9, 2019}{}{}
\item Conference for the Canadian Mathematical Society (CMS) consisting of 4 days of prize lectures, plenary speakers, scientific sessions and panels.
\end{rSubsection}

\begin{rSubsection}{CMS Winter Meeting Topological Mathematical Finance Mini-Course}{Dec 6, 2019}{}{}
\item  Introduction to the mathematics used in modern finance, and to the type of applications it finds in the world of quantitative analysts. We will touch on such mathematical techniques as the Ito calculus, stochastic control, and BSDE’s. These open up applications in finance, to topics such as derivative securities, hedging, risk neutrality, complete and incomplete markets, credit risk, volatility modelling, portfolio optimization, and optimal execution.
\end{rSubsection}

\begin{rSubsection}{CMS Winter Meeting Topological Data Analysis Mini-Course}{Dec 6, 2019}{}{}
\item Introduction to relevant topological ideas and approaches, as well as demonstrating how they can be used in data science to solve real practical problems.
\end{rSubsection}

\begin{rSubsection}{Graduate Professional Skills (GPS) program at UofT}{Oct 2018}{}{}
\item Covers various topics that range from tools for statistical and data analysis to strategies for academic publication and grant writing.
\end{rSubsection}

\begin{rSubsection}{Selected Areas in Cryptography (SAC) Summer School}{Aug 14--15 2017}{}{}
\item School for students to learn about symmetric key, public key, and post-quantum cryptography during Canada's research conference on cryptography.
\end{rSubsection}

\end{rSection}

\begin{rSection}{Relevant Graduate Courses}
\textbf{Computational Mathematics: Math models}\\
\textbf{Computational Mathematics: Numerical Methods}\\
 \textbf{Statistical Methods for Machine Learning and Data Mining}\\
 \textbf{Applied Stochastic Control: High Frequency and Algorithmic Trading}
\end{rSection}

\begin{rSection}{Relevant Undergraduate Courses}
\textbf{Applied Partial Differential Equations}\\
\textbf{Mathematical Modelling in Science }\\
 \textbf{Probability with Physical Applications}\\
 \textbf{Software Construction}
\end{rSection}

\end{document}
